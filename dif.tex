\documentclass[10pt,a4paper]{article}
\usepackage[utf8x]{inputenc}
\usepackage[L7x]{fontenc}
\usepackage[english]{babel}
\usepackage{amsmath}
\usepackage{amsfonts}
\usepackage{amssymb}
\usepackage{graphicx}
\usepackage{lmodern}
\begin{document}
Modelis:
\begin{itemize}
\item Sritis $\Omega$ – rutulys;
\item sritis $Q$, t.y.,    $\Omega$ aplinka – kubas su periodinėmis kraštinėmis sąlygomis;
\item pradiniu laiko momentu srityje $Q_\Omega \triangleq Q - \Omega$ turime $N_Q$ dalelių;
\item srityje $Q_\Omega$ dalelės juda pagal Brouno dėsnį (su nuliniu postūmiu ir tam tikru difuziniu koeficientu $v_Q$); 
\item nagrinėjama sistemos evoliucija  laiko momentais $t_0, t_1 \triangleq t_0 + \Delta t, ...$;
\item  dalelė pereina į $\Omega$ su tikimybe $p \triangleq f(N_\Omega, \Delta t)$. Tokiu atveju srityje   $Q_\Omega$ atsiranda dar viena dalelė; 
\item srityje $\Omega$ dalelės juda pagal Brouno dėsnį (su nuliniu postūmiu ir tam tikru difuziniu koeficientu $v_\Omega$) ir gali ,,dingti`` pagal tam tikrą dėsnį; 
\end{itemize}

\textbf{Uždavinys.}
Pagal keletą dalelių trajektorijų sukurti PDE (su nelokaliomis sąlygomis).




   
\end{document}